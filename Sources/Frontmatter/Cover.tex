%%%%%%%%%%%%%%%%%%%%%%%%%%%%%%%%%%%%%%%%%%%%%%%%%%%%%%%%%%%%%%%%%%%%%%%%%%%%%%%%%%%%%%%%%%%%%%%%%
%%% Modèle pour la 1ère de couverture des thèses préparées à l'Institut Polytechnique de Paris, basé sur le modèle produit par Guillaume BRIGOT / Template for front cover of thesis made at Institut Polytechnique de Paris, based on the template made by Guillaume BRIGOT
%%% Mis à jour par Aurélien ARNOUX (École polytechnique)/ Updated by Aurélien ARNOUX (École polytechnique)
%%% Nouvelle mise à jour par Jan SCHNEIDER (CPHT, École polytechnique)/ Further updated by Jan SCHNEIDER (CPHT, École polytechnique)
%%% Les instructions concernant chaque donnée à remplir sont données en bloc de commentaire / Rules to fill this file are given in comment blocks
%%% ATTENTION Ces informations doivent tenir sur une seule page une fois compilées / WARNING These informations must contain in no more than one page once compiled
%%%%%%%%%%%%%%%%%%%%%%%%%%%%%%%%%%%%%%%%%%%%%%%%%%%%%%%%%%%%%%%%%%%%%%%%%%%%%%%%%%%%%%%%%%%%%%%%


%%%%%%%%%%%%%%%%%%%%%%%%%%%%%%%%%%%%%%%%%%%%%%%%%%%%%%%%%%%%%%%%%%%%%%%
%%% Mise en page / Page layout
%%% NE RIEN MODIFIER EXCEPTÉ SI BESOIN / DO NOT MODIFY EXCEPT IF NEEDED
%%%%%%%%%%%%%%%%%%%%%%%%%%%%%%%%%%%%%%%%%%%%%%%%%%%%%%%%%%%%%%%%%%%%%%%%

\newgeometry{margin=1cm}



\begingroup\label{cover}
% \mycoverfont
\thispagestyle{empty}

% \color{black} 
\hfill \vfill \tiny \ecodocnum
\begin{textblock}{5}(0,0)
	\textblockcolour{black}
    \centering
    \vspace{8mm}
	\includegraphics[width=5cm]{graphics/frontmatter/ip-paris-white.pdf}
	\vspace{300mm}
\end{textblock}


\begin{textblock}{1}(0.6,3)
	\Large{\rotatebox{90}{\textcolor{white}{\textsf{\textbf{NNT~:~\NNT}}}}}
\end{textblock}

\begin{textblock}{1}(1,7)
	\rotatebox{90}{\textcolor{white}{{
		\fontsize{44pt}{44pt}\selectfont 
		% \mycoverthesefont
		\textsf{\textbf{Thèse de doctorat}}
		}}
	}
\end{textblock}        


\textblockcolour{white}
\begin{textblock}{1}(\hpostt,\vpostt)
	\textblockcolour{white}
	\logoEtt 
\end{textblock}

\begin{textblock}{1}(\hpos,\vpos)
	\textblockcolour{white}
	\logoEt
\end{textblock}

%\vspace{6cm}
%% Texte
\begin{textblock}{10}(5.5,3.5)
	\textblockcolour{white}
	
	% \color{black}
	%\begin{center}  
	\begin{flushright}
		\huge{\PhDTitleEN} \\ \bigskip %% Titre de la thèse 
		\vfill
		% \color{black} %% Couleur noire du reste du texte
		\normalsize {Thèse de doctorat de l'Institut Polytechnique de Paris} \\
		préparée à \PhDworkingplace \\ \bigskip
		\vfill
		École doctorale n\(^{\circ}\)\ecodocnum~\ecodoctitle~(\ecodocacro)  \\
		
		\small{Spécialité de doctorat: \PhDspeciality} \\ \bigskip %% Spécialité 
		\vfill  
		\footnotesize{Thèse présentée et soutenue à \defenseplace, le \defensedate, par} \\ \bigskip
		\vfill
		\Large{\textbf{\textsc{\PhDname}}} %% Nom du docteur
		\vfill
		%\bigskip
	\end{flushright}
	
	%\end{center}
	% \color{black}
	%% Jury
	\begin{flushleft}
		
		\small Composition du Jury~:
	\end{flushleft}
	%% Members of the jury

	\small
	%\begin{center}
	\newcolumntype{L}[1]{>{\raggedright\let\newline \\ \arraybackslash\hspace{0pt}}m{#1}}
	\newcolumntype{R}[1]{>{\raggedleft\let\newline \\ \arraybackslash\hspace{0pt}}lm{#1}}
	
	
    %% Update if members have been added or removed
	\begin{flushleft}\label{jury}
	\begin{tabular}{@{} L{9cm} R{4cm}}
		\jurynameA  \\ \juryadressA & \juryroleA \\[1em]
		\jurynameB  \\ \juryadressB & \juryroleB \\[1em]
		\jurynameC  \\ \juryadressC & \juryroleC \\[1em]
		\jurynameD  \\ \juryadressD & \juryroleD \\[1em]
		\jurynameE  \\ \juryadressE & \juryroleE \\[1em]
		\jurynameF  \\ \juryadressF & \juryroleF \\[1em]
		\jurynameG  \\ \juryadressG & \juryroleG \\[5pt]
		\jurynameH  \\ \juryadressH & \juryroleH \\[5pt]
	\end{tabular} 
	\end{flushleft}   
	%\end{center}
\end{textblock}
\endgroup
\clearpage

\restoregeometry
