
\newcommand{\PhDname}{Peter Pan}


\newcommand{\PhDTitleEN}{How to never grow up}	

\newcommand{\keywordsEN}{keyword 1, kw2, kw3}

\newcommand{\abstractEN}{
  Some abstract
}



\newcommand{\PhDTitleFR}{bleu, bli, blub}

\newcommand{\keywordsFR}{bleu, bli, blub}

\newcommand{\abstractFR}{
  Quelques mots en français.
}	


%% Numéro National de Thèse (donnée par la bibliothèque à la suite du 1er dépôt)/ National Thesis Number (given by the Library after the first deposit)
\newcommand{\NNT}{20XXIPPAXXXX}
%% Full name of Doctoral School : École doctorale de l'Institut Polytechnique de Paris, École doctorale de mathématiques Hadamard
\newcommand{\ecodoctitle}{de l'Institut Polytechnique de Paris}
%% Sigle de l'ED : EDIPP, EDMH / Acronym of the Doctoral School : EDIPP, EDMH
\newcommand{\ecodocacro}{EDIPP}
%% Doctoral School number : 626 (EDIPP), 574 (EDMH)
\newcommand{\ecodocnum}{626}
%% Spécialité de doctorat / Speciality 
\newcommand{\PhDspeciality}{Physique}
%% Établissement de préparation / PhD working place :  l'École polytechnique, l'École nationale supérieure de techniques avancées, l'École nationale de la statistique et de l’administration économique, Télécom ParisTech, Télécom SudParis, l’École des hautes études commerciales de Paris   
\newcommand{\PhDworkingplace}{l'École Polytechnique}
%%  Place of defense
\newcommand{\defenseplace}{Palaiseau}
%% Date de soutenance / Date of defense
\newcommand{\defensedate}{DATE}



\newcommand{\logoEtt}{\includegraphics[height=30mm]{graphics/frontmatter/logo_CPHT.pdf}}
%% Logo of laboratory or establishment.
\newcommand{\vpos}{0.2}
% If needed, modify to align logos vertilcally
\newcommand{\hpos}{13}
%%  If needed, modify to align logos horizontaly
\newcommand{\logoEt}{\includegraphics[height=50mm]{graphics/frontmatter/X_IPParis_blue.pdf}}
%% Institution logo. Filename correspond to institution acronym : ENSAE, ENSTA, TP, TSP, X
\newcommand{\vpostt}{0.33}
%% If needed, modify to align logos vertilcally
\newcommand{\hpostt}{5.5}
%% If needed, modify to align logos horizontaly


%%% JURY

% Choice of the jury's president is made during the defense. Thus, it must be specified only for the second file deposition in ADUM.
% All the jury members listed here must have been present during the defense.

%  Member n°1 (President)
\newcommand{\jurynameA}{Prénom Nom}
\newcommand{\juryadressA}{Statut, Établissement (Unité de recherche)}
\newcommand{\juryroleA}{Président}

%  Member n°2 (Reviewer)
\newcommand{\jurynameB}{Prénom Nom}
\newcommand{\juryadressB}{Statut, Établissement (Unité de recherche)}
\newcommand{\juryroleB}{Rapporteur}

%  Member n°3 (Reviewer)
\newcommand{\jurynameC}{Prénom Nom}
\newcommand{\juryadressC}{Statut, Établissement (Unité de recherche)}
\newcommand{\juryroleC}{Rapporteur}

%  Member n°4 (Examiner)
\newcommand{\jurynameD}{Prénom Nom}
\newcommand{\juryadressD}{Statut, Établissement (Unité de recherche)}
\newcommand{\juryroleD}{Examinateur}

%  Member n°5 (Thesis supervisor)
\newcommand{\jurynameE}{Prénom Nom}
\newcommand{\juryadressE}{Statut, Établissement (Unité de recherche)}
\newcommand{\juryroleE}{Directeur de thèse}

%  Member n°6 (Thesis co-supervisor)
\newcommand{\jurynameF}{Prénom Nom}
\newcommand{\juryadressF}{Statut, Établissement (Unité de recherche)}
\newcommand{\juryroleF}{Invité}

%  Member n°7 (Guest)
\newcommand{\jurynameG}{Prénom Nom}
\newcommand{\juryadressG}{Statut, Établissement (Unité de recherche)}
\newcommand{\juryroleG}{Invité}

%  Member n°8 (Guest)
\newcommand{\jurynameH}{Prénom Nom}
\newcommand{\juryadressH}{Statut, Établissement (Unité de recherche)}
\newcommand{\juryroleH}{Invité}

%  More jury members can be added according to the same model