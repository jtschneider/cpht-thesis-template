% !TEX root = Thesis.tex
\documentclass[a4paper,11pt]{mimosis}

\usepackage{metalogo}

%%%%%%%%%%%%%%%%%%%%%%%%%%%%%%%%%%%%%%%%%%%%%%%%%%%%%%%%%%%%%%%%%%%%%%%%
% Some of my favourite personal adjustments
%%%%%%%%%%%%%%%%%%%%%%%%%%%%%%%%%%%%%%%%%%%%%%%%%%%%%%%%%%%%%%%%%%%%%%%%
%
% These are the adjustments that I consider necessary for typesetting
% a nice thesis. However, they are *not* included in the template, as
% I do not want to force you to use them.

% This ensures that I am able to typeset bold font in table while still aligning the numbers
% correctly.

\usepackage{etoolbox}
\usepackage{physics}

% \usepackage{afterpage} %% for page specific geometry


% Make KOMA script happy?
% fore more details see documentatio of KOMA Script options on margins and textarea at
% mirrors.ctan.org/macros/latex/contrib/koma-script/doc/scrguien.pdf
\KOMAoptions{DIV=12}
\KOMAoptions{BCOR=12mm}


\usepackage{multicol}
\setlength{\columnseprule}{0pt}
\setlength\columnsep{10pt}

\usepackage[absolute,overlay]{textpos}


\newcommand{\PhDname}{Peter Pan}


\newcommand{\PhDTitleEN}{How to never grow up}	

\newcommand{\keywordsEN}{keyword 1, kw2, kw3}

\newcommand{\abstractEN}{
  Some abstract
}



\newcommand{\PhDTitleFR}{bleu, bli, blub}

\newcommand{\keywordsFR}{bleu, bli, blub}

\newcommand{\abstractFR}{
  Quelques mots en français.
}	


%% Numéro National de Thèse (donnée par la bibliothèque à la suite du 1er dépôt)/ National Thesis Number (given by the Library after the first deposit)
\newcommand{\NNT}{20XXIPPAXXXX}
%% Full name of Doctoral School : École doctorale de l'Institut Polytechnique de Paris, École doctorale de mathématiques Hadamard
\newcommand{\ecodoctitle}{de l'Institut Polytechnique de Paris}
%% Sigle de l'ED : EDIPP, EDMH / Acronym of the Doctoral School : EDIPP, EDMH
\newcommand{\ecodocacro}{EDIPP}
%% Doctoral School number : 626 (EDIPP), 574 (EDMH)
\newcommand{\ecodocnum}{626}
%% Spécialité de doctorat / Speciality 
\newcommand{\PhDspeciality}{Physique}
%% Établissement de préparation / PhD working place :  l'École polytechnique, l'École nationale supérieure de techniques avancées, l'École nationale de la statistique et de l’administration économique, Télécom ParisTech, Télécom SudParis, l’École des hautes études commerciales de Paris   
\newcommand{\PhDworkingplace}{l'École Polytechnique}
%%  Place of defense
\newcommand{\defenseplace}{Palaiseau}
%% Date de soutenance / Date of defense
\newcommand{\defensedate}{DATE}



\newcommand{\logoEtt}{\includegraphics[height=30mm]{graphics/frontmatter/logo_CPHT.pdf}}
%% Logo of laboratory or establishment.
\newcommand{\vpos}{0.2}
% If needed, modify to align logos vertilcally
\newcommand{\hpos}{13}
%%  If needed, modify to align logos horizontaly
\newcommand{\logoEt}{\includegraphics[height=50mm]{graphics/frontmatter/X_IPParis_blue.pdf}}
%% Institution logo. Filename correspond to institution acronym : ENSAE, ENSTA, TP, TSP, X
\newcommand{\vpostt}{0.33}
%% If needed, modify to align logos vertilcally
\newcommand{\hpostt}{5.5}
%% If needed, modify to align logos horizontaly


%%% JURY

% Choice of the jury's president is made during the defense. Thus, it must be specified only for the second file deposition in ADUM.
% All the jury members listed here must have been present during the defense.

%  Member n°1 (President)
\newcommand{\jurynameA}{Prénom Nom}
\newcommand{\juryadressA}{Statut, Établissement (Unité de recherche)}
\newcommand{\juryroleA}{Président}

%  Member n°2 (Reviewer)
\newcommand{\jurynameB}{Prénom Nom}
\newcommand{\juryadressB}{Statut, Établissement (Unité de recherche)}
\newcommand{\juryroleB}{Rapporteur}

%  Member n°3 (Reviewer)
\newcommand{\jurynameC}{Prénom Nom}
\newcommand{\juryadressC}{Statut, Établissement (Unité de recherche)}
\newcommand{\juryroleC}{Rapporteur}

%  Member n°4 (Examiner)
\newcommand{\jurynameD}{Prénom Nom}
\newcommand{\juryadressD}{Statut, Établissement (Unité de recherche)}
\newcommand{\juryroleD}{Examinateur}

%  Member n°5 (Thesis supervisor)
\newcommand{\jurynameE}{Prénom Nom}
\newcommand{\juryadressE}{Statut, Établissement (Unité de recherche)}
\newcommand{\juryroleE}{Directeur de thèse}

%  Member n°6 (Thesis co-supervisor)
\newcommand{\jurynameF}{Prénom Nom}
\newcommand{\juryadressF}{Statut, Établissement (Unité de recherche)}
\newcommand{\juryroleF}{Invité}

%  Member n°7 (Guest)
\newcommand{\jurynameG}{Prénom Nom}
\newcommand{\juryadressG}{Statut, Établissement (Unité de recherche)}
\newcommand{\juryroleG}{Invité}

%  Member n°8 (Guest)
\newcommand{\jurynameH}{Prénom Nom}
\newcommand{\juryadressH}{Statut, Établissement (Unité de recherche)}
\newcommand{\juryroleH}{Invité}

%  More jury members can be added according to the same model

% DELETE THIS FOR use in actual thesis. This just creates filler text
\usepackage{lipsum}

%%%%%%%%%%%%%%%%%%%%%%%%%%
%% GEOMETRY
%%%%%%%%%%%%%%%%%%%%%%%%%%
\usepackage{geometry}
\geometry{
	inner=25mm,
	top=25mm,
	outer=30mm,
	bottom=25mm
}



%%%%%%%%%%%%%%%%%%%%%%%%%%%%%%%%%%%%%%%%%%%%%%%%%%%%%%%%%%%%%%%%%%%%%%%%
% Hyperlinks & bookmarks
%%%%%%%%%%%%%%%%%%%%%%%%%%%%%%%%%%%%%%%%%%%%%%%%%%%%%%%%%%%%%%%%%%%%%%%%

\usepackage[%
  colorlinks = true,
  citecolor  = BrickRed,
  linkcolor  = RoyalBlue,
  urlcolor   = RoyalBlue,
  unicode,
  pdftitle={\PhDTitleEN},
  pdfauthor={\PhDname},
  pdfsubject={Manuscrit de thèse de doctorat},
  ]{hyperref}

\usepackage{bookmark}
\usepackage[capitalise]{cleveref}

%%%%%%%%%%%%%%%%%%%%%%%%%%%%%%%%%%%%%%%%%%%%%%%%%%%%%%%%%%%%%%%%%%%%%%%%
% Bibliography
%%%%%%%%%%%%%%%%%%%%%%%%%%%%%%%%%%%%%%%%%%%%%%%%%%%%%%%%%%%%%%%%%%%%%%%%
%
% I like the bibliography to be extremely plain, showing only a numeric
% identifier and citing everything in simple brackets. The first names,
% if present, will be initialized. DOIs and URLs will be preserved.

\usepackage[%
  autocite     = plain,
  backend      = biber,
  % doi          = true,
  eprint       = true,
  url          = false,
  giveninits   = true,
  hyperref     = true,
  maxbibnames  = 99,
  maxcitenames = 99,
  sorting      = none,
  % sortcites    = true,
  % sortsets     = true,
  style        = phys,
  % citestyle    = authoryear-comp,
  % bibstyle     = phys,
  backref,%
  natbib       = true,
  ]{biblatex}

\input{bibliography-mimosis}
\addbibresource{Thesis.bib}



%%%%%%%%%%%%%%%%%%%%%%%%%%%%%%%%%%%%%%%%%%%%%%%%%%%%%%%%%%%%%%%%%%%%%%%%
% Fonts
%%%%%%%%%%%%%%%%%%%%%%%%%%%%%%%%%%%%%%%%%%%%%%%%%%%%%%%%%%%%%%%%%%%%%%%%

\ifxetexorluatex
  \usepackage{unicode-math}
  \usepackage{fontspec}

  % \setmainfont{Garamond}
  % \setmainfont{EB Garamond}
  % \setmathfont{Garamond Math}


  % STIX Math Font
  % \setmathfont{STIXTwoMath-Regular}[
  % Extension={.otf},
  % Path=./fonts/STIX2/,
  % Scale=1]

  % \setmainfont{STIXTwoText}[
  %   Extension={.otf},
  %   Path=./fonts/STIX2/,
  %   UprightFont={*-Regular},
  %   BoldFont={*-Bold},
  %   ItalicFont={*-Italic},
  %   BoldItalicFont={*-BoldItalic}
  % ]
  % \setsansfont{texgyreheros}[
  %   Extension={.otf},
  %   Path=./fonts/Heros/,
  %   UprightFont={*-regular},
  %   BoldFont={*-bold},
  %   ItalicFont={*-italic},
  %   BoldItalicFont={*-bolditalic}
  % ]


  % Classical LaTeX Fonts "Latin Modern"
  \setmathfont{latinmodern-math}[
    Extension={.otf},
    Path=./fonts/Latin_Modern/,
    Scale=1
  ]

  \setmainfont{lmroman}[
    Extension={.otf},
    Path=./fonts/Latin_Modern/,
    UprightFont={*-regular},
    BoldFont={*-bold},
    ItalicFont={*-italic},
    BoldItalicFont={*-bolditalic},
    FontFace={db}{n}{Font=*demi-regular,Ligatures=CommonOff},
    FontFace={db}{it}{Font=*demi-oblique,Ligatures=CommonOff},
    FontFace={sc}{n}{Font=*caps-regular,Ligatures=CommonOff},
    FontFace={sc}{it}{Font=*caps-oblique,Ligatures=CommonOff},
  ]

  \setmonofont{lmmono}[
    Scale=MatchLowercase,
    Extension={.otf},
    Path=./fonts/Latin_Modern/,
    UprightFont={*-regular},
    % BoldFont={*-bold},
    ItalicFont={*-italic}
  ]

  \setsansfont{lmsans}[
    Extension={.otf},
    Path=./fonts/Latin_Modern/,
    UprightFont={*-regular},
    BoldFont={*-bold},
    ItalicFont={*-oblique},
    BoldItalicFont={*-boldoblique}
  ]

  % \setfontfamily\mycoverthesefont{Libre Franklin}
  % \setfontfamily\mycoverthesefont{IBM Plex Sans}
  % \setfontfamily\mycoverfont{STIXTwoText}


  % Set fonts for sections, chapters, and page headers
  % options are:
  % normal (main) Font
  % font series: regular, bold, SmallCaps,...
  % font shape: normal, italic, oblique,...
  % \setkomafont{sectioning}{\normalfont\fontseries{db}\fontshape{n}}
  \setkomafont{sectioning}{\normalfont\fontseries{db}\fontshape{it}}
  \setkomafont{pagehead}{\normalfont\fontseries{sc}\fontshape{n}}

\else
  % \usepackage[lf]{ebgaramond}
  % \usepackage{stix2}
  \usepackage{lmodern}
  % \usepackage[garamondx,cmbraces]{newtxmath}
  % \usepackage[oldstyle,scale=0.7]{sourcecodepro}
  \singlespacing
  \usepackage{amsmath}
  \usepackage{amssymb}
  \usepackage{mathtools}
\fi


%%%%% Uncomment for Glossary anc acronyms

\newacronym[description={Singular value decomposition}]{SVD}{SVD}{singular value decomposition}
\newacronym{SNF}{SNF}{Smith normal form}
% \newacronym[description={Topological data analysis}]{TDA}{TDA}{topological data analysis}

\newglossaryentry{LaTeX}{%
  name        = {\LaTeX},
  description = {A document preparation system},
  sort        = {LaTeX},
}

\newglossaryentry{Real numbers}{%
  name        = {\(\mathfrak{R}\)},
  description = {The set of real numbers},
  sort        = {Real numbers},
}

\makeindex
\makeglossaries

%%%%%%%%%%%%%%%%%%%%%%%%%%%%%%%%%%%%%%%%%%%%%%%%%%%%%%%%%%%%%%%%%%%%%%%%
% Ordinals
%%%%%%%%%%%%%%%%%%%%%%%%%%%%%%%%%%%%%%%%%%%%%%%%%%%%%%%%%%%%%%%%%%%%%%%%

\makeatletter
\@ifundefined{st}{%
  \newcommand{\st}{\textsuperscript{\textup{st}}\xspace}
}{}
\@ifundefined{rd}{%
  \newcommand{\rd}{\textsuperscript{\textup{rd}}\xspace}
}{}
\@ifundefined{nd}{%
  \newcommand{\nd}{\textsuperscript{\textup{nd}}\xspace}
}{}
\makeatother

\renewcommand{\th}{\textsuperscript{\textup{th}}\xspace}


%%%%%%%%%%%%%%%%%%%%%%%%%%%%%%%%%%%%%%%%%%%%%%%%%%%%%%%%%%%%%%%%%%%%%%%%
% Incipit
%%%%%%%%%%%%%%%%%%%%%%%%%%%%%%%%%%%%%%%%%%%%%%%%%%%%%%%%%%%%%%%%%%%%%%%%

\title{\PhDTitleEN}
% \subtitle{}
\author{\PhDname}

\begin{document}

%%%% Template files for a Thesis at the CPHT.
%%%% Configure the content of the following frontmatter in Sources/Frontmatter/configuration.tex
\frontmatter
  %%%%%%%%%%%%%%%%%%%%%%%%%%%%%%%%%%%%%%%%%%%%%%%%%%%%%%%%%%%%%%%%%%%%%%%%%%%%%%%%%%%%%%%%%%%%%%%%%
%%% Modèle pour la 1ère de couverture des thèses préparées à l'Institut Polytechnique de Paris, basé sur le modèle produit par Guillaume BRIGOT / Template for front cover of thesis made at Institut Polytechnique de Paris, based on the template made by Guillaume BRIGOT
%%% Mis à jour par Aurélien ARNOUX (École polytechnique)/ Updated by Aurélien ARNOUX (École polytechnique)
%%% Nouvelle mise à jour par Jan SCHNEIDER (CPHT, École polytechnique)/ Further updated by Jan SCHNEIDER (CPHT, École polytechnique)
%%% Les instructions concernant chaque donnée à remplir sont données en bloc de commentaire / Rules to fill this file are given in comment blocks
%%% ATTENTION Ces informations doivent tenir sur une seule page une fois compilées / WARNING These informations must contain in no more than one page once compiled
%%%%%%%%%%%%%%%%%%%%%%%%%%%%%%%%%%%%%%%%%%%%%%%%%%%%%%%%%%%%%%%%%%%%%%%%%%%%%%%%%%%%%%%%%%%%%%%%


%%%%%%%%%%%%%%%%%%%%%%%%%%%%%%%%%%%%%%%%%%%%%%%%%%%%%%%%%%%%%%%%%%%%%%%
%%% Mise en page / Page layout
%%% NE RIEN MODIFIER EXCEPTÉ SI BESOIN / DO NOT MODIFY EXCEPT IF NEEDED
%%%%%%%%%%%%%%%%%%%%%%%%%%%%%%%%%%%%%%%%%%%%%%%%%%%%%%%%%%%%%%%%%%%%%%%%

\newgeometry{margin=1cm}



\begingroup\label{cover}
% \mycoverfont
\thispagestyle{empty}

% \color{black} 
\hfill \vfill \tiny \ecodocnum
\begin{textblock}{5}(0,0)
	\textblockcolour{black}
    \centering
    \vspace{8mm}
	\includegraphics[width=5cm]{graphics/frontmatter/ip-paris-white.pdf}
	\vspace{300mm}
\end{textblock}


\begin{textblock}{1}(0.6,3)
	\Large{\rotatebox{90}{\textcolor{white}{\textsf{\textbf{NNT~:~\NNT}}}}}
\end{textblock}

\begin{textblock}{1}(1,7)
	\rotatebox{90}{\textcolor{white}{{
		\fontsize{44pt}{44pt}\selectfont 
		% \mycoverthesefont
		\textsf{\textbf{Thèse de doctorat}}
		}}
	}
\end{textblock}        


\textblockcolour{white}
\begin{textblock}{1}(\hpostt,\vpostt)
	\textblockcolour{white}
	\logoEtt 
\end{textblock}

\begin{textblock}{1}(\hpos,\vpos)
	\textblockcolour{white}
	\logoEt
\end{textblock}

%\vspace{6cm}
%% Texte
\begin{textblock}{10}(5.5,3.5)
	\textblockcolour{white}
	
	% \color{black}
	%\begin{center}  
	\begin{flushright}
		\huge{\PhDTitleEN} \\ \bigskip %% Titre de la thèse 
		\vfill
		% \color{black} %% Couleur noire du reste du texte
		\normalsize {Thèse de doctorat de l'Institut Polytechnique de Paris} \\
		préparée à \PhDworkingplace \\ \bigskip
		\vfill
		École doctorale n\(^{\circ}\)\ecodocnum~\ecodoctitle~(\ecodocacro)  \\
		
		\small{Spécialité de doctorat: \PhDspeciality} \\ \bigskip %% Spécialité 
		\vfill  
		\footnotesize{Thèse présentée et soutenue à \defenseplace, le \defensedate, par} \\ \bigskip
		\vfill
		\Large{\textbf{\textsc{\PhDname}}} %% Nom du docteur
		\vfill
		%\bigskip
	\end{flushright}
	
	%\end{center}
	% \color{black}
	%% Jury
	\begin{flushleft}
		
		\small Composition du Jury~:
	\end{flushleft}
	%% Members of the jury

	\small
	%\begin{center}
	\newcolumntype{L}[1]{>{\raggedright\let\newline \\ \arraybackslash\hspace{0pt}}m{#1}}
	\newcolumntype{R}[1]{>{\raggedleft\let\newline \\ \arraybackslash\hspace{0pt}}lm{#1}}
	
	
    %% Update if members have been added or removed
	\begin{flushleft}\label{jury}
	\begin{tabular}{@{} L{9cm} R{4cm}}
		\jurynameA  \\ \juryadressA & \juryroleA \\[1em]
		\jurynameB  \\ \juryadressB & \juryroleB \\[1em]
		\jurynameC  \\ \juryadressC & \juryroleC \\[1em]
		\jurynameD  \\ \juryadressD & \juryroleD \\[1em]
		\jurynameE  \\ \juryadressE & \juryroleE \\[1em]
		\jurynameF  \\ \juryadressF & \juryroleF \\[1em]
		\jurynameG  \\ \juryadressG & \juryroleG \\[5pt]
		\jurynameH  \\ \juryadressH & \juryroleH \\[5pt]
	\end{tabular} 
	\end{flushleft}   
	%\end{center}
\end{textblock}
\endgroup
\clearpage

\restoregeometry

  \newgeometry{margin=3cm}
\begin{titlepage}
  \vspace*{3cm}
  \makeatletter
  \begin{center}
    \begin{huge}
      \@title
    \end{huge}\\[1.15em]
    %
    \ifdefined \r@subtitle
    \begin{Large}
      \@subtitle
    \end{Large}\\[0.6em]
    \fi
    %
    \emph{by}\\[0.6em]
    \@author
    %
    % \vfill
    % \begin{center}
    %   \includegraphics[height=10cm]{graphics/frontmatter/X_IPParis_blue.pdf}
    % \end{center}
    %
    \vfill
    
    {\large Thèse de doctorat de l'Institut Polytechnique de Paris\\
    préparée au Centre de Physique Théorique\\
    à\\
    \textsc{l'École Polytechnique}}
   
  \end{center}
  \makeatother
\end{titlepage}

\restoregeometry

\newpage
\null
\thispagestyle{empty}
\newpage

  
%%%%%%%%%%%%%%%%%%%%%%%%%%%%%%%%%%%%%%%%%%%%%%%%%%%%%%%%%%%%%%%%%%%%%%%%%%%%%%%%%%%%%%%%%%%%%%%%%%%%%%%%%%%%%%%%%%%%%%%%%%%%%%%%%%%%%%%%%%%%%%%%%%%%%%%%%%%%%%%%%%%%%%
%%% Modèle pour la 4ème de couverture des thèses préparées à l'Institut Polytechnique de Paris, basé sur le modèle produit par Nikolas STOTT 
%%% / Template for back cover of thesis made at Institut Polytechnique de Paris, based on the template made by Nikolas STOTT
%%% Mis à jour par Aurélien ARNOUX (École polytechnique)/ Updated by Aurélien ARNOUX (École polytechnique)
%%% Nouvelle mise à jour par Jan SCHNEIDER (CPHT, École polytechnique)/ Updated by Jan SCHNEIDER (CPHT, École polytechnique)
%%% ATTENTION Ces informations doivent tenir sur une seule page une fois compilées /
%%% /WARNING These informations must contain in no more than one page once compiled
%%%%%%%%%%%%%%%%%%%%%%%%%%%%%%%%%%%%%%%%%%%%%%%%%%%%%%%%%%%%%%%%%%%%%%%%%%%%%%%%%%%%%%%%%%%%%%%%%%%%%%%%%%%%%%%%%%%%%%%%%%%%%%%%%%%%%%%%%%%%%%%%%%%%%%%%%%%%%%%%%%%%%%%
%%% Version du 5 mai 2022 : utilisation de .pdf au lieu de .png pour les logos
%%%%%%%%%%%%%%%%%%%%%%%%%%%%%%%%%%%%%%%%%%%%%%%%%%%%%%%%%%%%%%%%%%%%%%%%%%%%%%%%%%%%%%%%%%%%%%%%%%%%%%%%%%%%%%%%%%%%%%%%%%%%%%%%%%%%%%%%%%%%%%%%%%%%%%%%%%%%%%%%%%%%%%%


\thispagestyle{empty}

% \begin{center}
%   \textsc{Abstract}
% \end{center}

% \vspace{1cm}
%
% \noindent
%

\newgeometry{top=15mm,bottom=20mm,left=15mm,right=15mm}


\begin{minipage}{60mm}
  \includegraphics[width=\textwidth]{graphics/frontmatter/ed-ip-paris.pdf}
\end{minipage}

\vspace*{1cm}

%%%Titre de la thèse en anglais / Thesis title in english
\begin{center}
\fcolorbox{black}{white}{\parbox{160mm}{
  \textbf{Title:} \PhDTitleEN 

  \medskip

  %%%Mots clés en anglais, séprarés par des ; / Keywords in english, separated by ;
  \textbf{Keywords:}  \keywordsEN %%3 à 6 mots clés%%
  \vspace{-2mm}
  \begin{multicols}{2}
    
  %%% Résumé en anglais / abstract in english
  \textbf{Abstract:} 
  \abstractEN
  \end{multicols}
}}
\end{center}

\vspace*{5mm}

%%%Titre de la thèse en français / Thesis title in french
\begin{center}
  \fcolorbox{black}{white}{\parbox{160mm}{
    \textbf{Titre:} \PhDTitleFR 
    \medskip
    
    %%%Mots clés en français, séprarés par des ; / Keywords in french, separated by ;
    \textbf{Mots clés:} \keywordsFR 
    \vspace{-2mm}
    
    %%% Résumé en français / abstract in french
    \begin{multicols}{2}
    \textbf{Résumé:} 
    \abstractFR 
    \end{multicols}
  }}
\end{center}
  
% \vspace*{0mm}
\vfill


\begin{minipage}[b][][l]{60mm}
  \textbf{Institut Polytechnique de Paris}\\
  91120 Palaiseau, Frances
\end{minipage}
% 
\hfill
% 
\begin{minipage}[b][][r]{22mm}
  \includegraphics[width=\textwidth]{graphics/frontmatter/ip-paris-petit.pdf}
\end{minipage}

\restoregeometry

  \selectlanguage{english}
  \tableofcontents

\mainmatter

%%%%%%%%%%%%% If you want to Partition your thesis, use the \part command like;
  \part[A good part]{%
    A good part\\
    %
    \vspace{1cm}
    %
    \begin{minipage}[l]{\textwidth}
    %
    \textnormal{%
      \normalsize
      %
      \begin{singlespace*}
        \onehalfspacing
        %
        You can also use parts in order to partition your great work
        into larger `chunks'. This involves some manual adjustments in
        terms of the layout, though.
      \end{singlespace*}
    }
    \end{minipage}
  }

  \include{Sources/Chapters/template.tex}

% This ensures that the subsequent sections are being included as root
% items in the bookmark structure of your PDF reader.
\bookmarksetup{startatroot}
\backmatter

  %%%%% Glossary %%%%%
  \begingroup
    \let\clearpage\relax
    \glsaddall
    \printglossary[type=\acronymtype]
    \newpage
    \printglossary
  \endgroup

  %%% Bibliography %%%%
  \printindex
  \printbibliography

  %%%%%%% !!!! The contents of this pages must fit in a single A4 page !!!!

\newgeometry{
left=16mm,
top=27mm,
right=16mm,
bottom=27mm
}


\phantomsection{}
\addcontentsline{toc}{chapter}{\protect\numberline{}Back cover}

% 
\thispagestyle{empty}

\fontsize{10}{10}\selectfont
\setlength{\columnseprule}{0pt}
\setlength\columnsep{10pt}

% \begin{textblock*}{61mm}(16mm,3mm)

\vspace*{-16mm}

\noindent\includegraphics[height=24mm]{graphics/frontmatter/ed-ip-paris.pdf}
% \end{textblock*}




\selectlanguage{french}
%%%Titre de la thèse en français / Thesis title in french
\begin{center}
	\fcolorbox{black}{white}{\parbox{0.98\textwidth}{
	\textbf{Titre:} \PhDTitleFR{} 
	\medskip

	%%%Mots clés en français, séprarés par des ;
	\textbf{Mots clés:} \keywordsFR{} 
	\vspace{-3mm}

	%%% Résumé en français / abstract in french
	\begin{multicols}{2}
	\textbf{Résumé:}
  %%%%%%% !!!! The contents of this pages must fit in a single A4 page !!!!
    \lipsum[1-2]
	\end{multicols}
  \vspace{-3.5mm}
	}}
	\end{center}

\vspace*{0mm}


\selectlanguage{british}
%%Titre de la thèse en anglais / Thesis title in english
\begin{center}
  \fcolorbox{black}{white}{\parbox{0.98\textwidth}{
  \textbf{Title:} \PhDTitleEN{}
  
  \medskip
  
  %%% Keywords in english, separated by ;
  \textbf{Keywords:}  \keywordsEN{}
  \vspace{-3mm}
  \begin{multicols}{2}
  \textbf{Abstract:}
    %%%%%%% !!!! The contents of this pages must fit in a single A4 page !!!!
    \lipsum[1-2]
  \end{multicols}
  \vspace{-3.5mm}
  }}
\end{center}



\hfill
\vfill

\noindent
\begin{minipage}{0.5\linewidth}
  \fontsize{11}{12}\selectfont
  \noindent \textbf{Institut Polytechnique de Paris}\\
  \noindent 91120 Palaiseau, France 
\end{minipage}
\hfill
\begin{minipage}{22mm}
\includegraphics[width=\linewidth]{graphics/frontmatter/ip-paris-petit.pdf}
\end{minipage}

\end{document}

\end{document}
